\chapter{Fazit} % (fold)
\label{cha:fazit}

Die Kartierung der Räume der Hochschule durch einen mobilen Roboter bietet einige Vorteile, da der Roboter eine aktuelle Lage der Räume aufzeigen kann. Auch ist das Thema außerhalb der Hochschule in vielen Robotikberreichen aktuell. Der Einsatz von Kartierung ist unter anderm in Katastrophengebieten von Interesse, da die dort vorherschende Lage von äußeren Faktoren verändert wurde und somit früher angefertigte Karten nicht mehr aktuell sind.\par
Die während der Lehrveranstaltung erstellten Karten sind von variabler Qualität, der Grundriss der Räume ist aber meist gut zu erkennen. Der Algorithmus scheitert oft an Räumen die voll gestellt sind, bzw. in denen nur schmale Wege zwischen den Räumen vorhanden sind. Auch ist auffallend das der Algorithmus bei gleichen Parametern bei der Ausrichtung der Räume in manchen Durchläufen scheitert und in anderen die Räume richtig positioniert. Der Akkueinsatz vereinfacht die Kartierung um ein vielfaches, da nicht mehr auf die Kabel und den Tisch geachtet werden muss. Es wäre wünschenswert wenn der Ladestand des Akkus während des Betriebs auf dem Rechner oder durch einen Warnton bei baldiger Lehrung angezeigt werden würde.\par
Wenn bei der Kartierung ausreichend langsam gefahren und auf Rückwärtsfahren verzichtet wurde, ist die Wahrscheinlichkeit hoch, dass bei mehrmaliger Duchführung des Algorithmus brauchbare Karten entstehen.\par
Im Allgemeinen wäre es von Interesse die Kartierung auf andere Berreiche wie die Wlan-Feldstärke zu erweitern. Auch wäre das Erstellen von 3D-Karten der Räume durch die Kinect von Interesse.

% chapter fazit (end)
\chapter{Einleitung} % (fold)
\label{cha:einleitung}

Ein großes Problem für Roboter ist es, dass sie sich in ihrer eigenen Umgebung zurechtfinden und anhand von Merkmalen der Umgebung ihre eigene Position erkennen müssen. Ein gutes Beispiel hierfür sind Staubsaugerroboter. Diese fahren meist von einer Basisstation los und erkunden die Umgebung. Hat der Roboter dann seine Aufgabe erledigt, fährt er wieder zu seiner Basisstation zurück. Um Objekte, wie z.\,B. die Basisstation, in einer Welt wiederfinden zu können, muss diese zunächst kartiert werden. Mit einer möglichen Lösung für die Kartierung einer Umgebung beschäftigt sich diese Arbeit. Konkret geht es dabei um den Pioneer 3-DX Roboter, der mit Hilfe eines Laserscanners die Umgebung kartiert. Im Unterschied zum Staubsaugerroboter geschieht dies allerdings nicht autonom, sondern mittels eines Joysticks. Der Roboter verwendet dabei diverse Pakete des \ac{ROS} um zunächst zu Kartieren und eine Karte der Umgebung zu erhalten.\par
In der vorliegenden Projektarbeit werden die Grundlagen des \ac{ROS}, das Vorgehen bei der Kartierung sowie beim Abspielen fertiger Karten beschrieben. Das Ergebnis der vorliegenden Arbeit sind Anleitungen für die ge\-nann\-ten Vorgänge, Informationen über alles was man dabei beachten sollte um Probleme zu vermeiden, sowie Beispiele in Form zwei unterschiedlicher Kartierungen.

% chapter einleitung (end)
% -------------------------------------------------------
% Daten für die Arbeit
% Wenn hier alles korrekt eingetragen wurde, wird das Titelblatt
% automatisch generiert. D.h. die Datei titelblatt.tex muss nicht mehr
% angepasst werden.

\newcommand{\hsmasprache}{de} % de oder en für Deutsch oder Englisch
% Für korrekt sortierte Literatureinträge, noch preambel.tex anpassen
% und zwar bei \usepackage[main=ngerman, english]{babel},
% \usepackage[pagebackref=false,german]{hyperref}
% und \usepackage[autostyle=true,german=quotes]{csquotes}

% Titel der Arbeit auf Deutsch
\newcommand{\hsmatitelde}{Kartierung eines Stockwerkes mittels GMapping}

% Titel der Arbeit auf Englisch
\newcommand{\hsmatitelen}{Application of a flux compensator for timetravel with a maximum velocity of warp~7}

% Weitere Informationen zur Arbeit
\newcommand{\hsmaort}{Mannheim}    % Ort
\newcommand{\hsmaautorvname}{Max} % Vorname(n)
\newcommand{\hsmaautornname}{Mustermann} % Nachname(n)
\newcommand{\hsmadatum}{20.07.18} % Datum der Abgabe
\newcommand{\hsmajahr}{2018} % Jahr der Abgabe
\newcommand{\hsmafirma}{Paukenschlag GmbH, Mannheim} % Firma bei der die Arbeit durchgeführt wurde
\newcommand{\hsmabetreuer}{Prof. Thomas Ihme, Hochschule Mannheim} % Betreuer an der Hochschule
\newcommand{\hsmazweitkorrektor}{Prof. Thomas Ihme, Hochschule Mannheim} % Betreuer im Unternehmen oder Zweitkorrektor
\newcommand{\hsmafakultaet}{I} % I für Informatik
\newcommand{\hsmastudiengang}{IB} % IB IMB UIB IM MTB

% Zustimmung zur Veröffentlichung
\setboolean{hsmapublizieren}{false}   % Einer Veröffentlichung wird zugestimmt
\setboolean{hsmasperrvermerk}{false} % Die Arbeit hat keinen Sperrvermerk

% -------------------------------------------------------
% Abstract

% Kurze (maximal halbseitige) Beschreibung, worum es in der Arbeit geht auf Deutsch
\newcommand{\hsmaabstractde}{Die Verwendung von Robotern in der Industrie ist schon seit einiger Zeit der Stand\-ard. Mittlerweile treten Roboter immer mehr im alltägliche Leben auf, da sie immer effizienter, zuverlässiger werden. Immer mehr geht es dabei auch um Roboter, die sich nicht nur einer Aufgabe widmen, sondern fähig sind viele verschiedene Dinge zu erledigen. Die Roboter sollen irgendwann gänzlich autonom arbeiten. Ein generelles Problem auf diesem Weg ist, dass Roboter die Welt nicht wie der Mensch durch Augen, sondern einzig durch seine Sensoren sieht. Man spricht hierbei vom \ac{SLAM}-Problem, bei dem der Roboter selbst eine Karte der Umgebung erstellt, gleichzeitig aber auch jederzeit wissen muss, wo er sich gerade befindet. Diese Arbeit beschäftigt sich mit diesem Problem. Das Ergebnis ist ein Prozess zum Kartieren einer Umgebung mit Hilfe eines Pioneer 3-DX Roboters.}

% Kurze (maximal halbseitige) Beschreibung, worum es in der Arbeit geht auf Englisch

\newcommand{\hsmaabstracten}{The European languages are members of the same family. Their separate existence is a myth. For science, music, sport, etc, Europe uses the same vocabulary. The languages only differ in their grammar, their pronunciation and their most common words. Everyone realizes why a new common language would be desirable: one could refuse to pay expensive translators. To achieve this, it would be necessary to have uniform grammar, pronunciation and more common words. If several languages coalesce, the grammar of the resulting language is more simple and regular than that of the individual languages. The new common language will be more simple and regular than the existing European languages. It will be as simple as Occidental; in fact, it will be Occidental. To an English person, it will seem like simplified English, as a skeptical Cambridge friend of mine told me what Occidental is.}
